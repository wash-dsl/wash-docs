\href{../gmock_for_dummies.md#actions-what-should-it-do}{\tt {\bfseries Actions}} specify what a mock function should do when invoked. This page lists the built-\/in actions provided by Google\+Test. All actions are defined in the {\ttfamily \+::testing} namespace.

\subsection*{Returning a Value}

\tabulinesep=1mm
\begin{longtabu} spread 0pt [c]{*{2}{|X[-1]}|}
\hline
\rowcolor{\tableheadbgcolor}\textbf{ Action  }&\textbf{ Description   }\\\cline{1-2}
\endfirsthead
\hline
\endfoot
\hline
\rowcolor{\tableheadbgcolor}\textbf{ Action  }&\textbf{ Description   }\\\cline{1-2}
\endhead
{\ttfamily Return()}  &Return from a {\ttfamily void} mock function.   \\\cline{1-2}
{\ttfamily Return(value)}  &Return {\ttfamily value}. If the type of {\ttfamily value} is different to the mock function\textquotesingle{}s return type, {\ttfamily value} is converted to the latter type {\itshape at the time the expectation is set}, not when the action is executed.   \\\cline{1-2}
{\ttfamily Return\+Arg$<$N$>$()}  &Return the {\ttfamily N}-\/th (0-\/based) argument.   \\\cline{1-2}
{\ttfamily Return\+New$<$T$>$(a1, ..., ak)}  &Return {\ttfamily new T(a1, ..., ak)}; a different object is created each time.   \\\cline{1-2}
{\ttfamily Return\+Null()}  &Return a null pointer.   \\\cline{1-2}
{\ttfamily Return\+Pointee(ptr)}  &Return the value pointed to by {\ttfamily ptr}.   \\\cline{1-2}
{\ttfamily Return\+Ref(variable)}  &Return a reference to {\ttfamily variable}.   \\\cline{1-2}
{\ttfamily Return\+Ref\+Of\+Copy(value)}  &Return a reference to a copy of {\ttfamily value}; the copy lives as long as the action.   \\\cline{1-2}
{\ttfamily Return\+Round\+Robin(\{a1, ..., ak\})}  &Each call will return the next {\ttfamily ai} in the list, starting at the beginning when the end of the list is reached.   \\\cline{1-2}
\end{longtabu}


\subsection*{Side Effects}

\tabulinesep=1mm
\begin{longtabu} spread 0pt [c]{*{2}{|X[-1]}|}
\hline
\rowcolor{\tableheadbgcolor}\textbf{ Action  }&\textbf{ Description   }\\\cline{1-2}
\endfirsthead
\hline
\endfoot
\hline
\rowcolor{\tableheadbgcolor}\textbf{ Action  }&\textbf{ Description   }\\\cline{1-2}
\endhead
{\ttfamily Assign(\&variable, value)}  &Assign {\ttfamily value} to variable.   \\\cline{1-2}
{\ttfamily Delete\+Arg$<$N$>$()}  &Delete the {\ttfamily N}-\/th (0-\/based) argument, which must be a pointer.   \\\cline{1-2}
{\ttfamily Save\+Arg$<$N$>$(pointer)}  &Save the {\ttfamily N}-\/th (0-\/based) argument to {\ttfamily $\ast$pointer}.   \\\cline{1-2}
{\ttfamily Save\+Arg\+Pointee$<$N$>$(pointer)}  &Save the value pointed to by the {\ttfamily N}-\/th (0-\/based) argument to {\ttfamily $\ast$pointer}.   \\\cline{1-2}
{\ttfamily Set\+Arg\+Referee$<$N$>$(value)}  &Assign {\ttfamily value} to the variable referenced by the {\ttfamily N}-\/th (0-\/based) argument.   \\\cline{1-2}
{\ttfamily Set\+Arg\+Pointee$<$N$>$(value)}  &Assign {\ttfamily value} to the variable pointed by the {\ttfamily N}-\/th (0-\/based) argument.   \\\cline{1-2}
{\ttfamily Set\+Argument\+Pointee$<$N$>$(value)}  &Same as {\ttfamily Set\+Arg\+Pointee$<$N$>$(value)}. Deprecated. Will be removed in v1.\+7.\+0.   \\\cline{1-2}
{\ttfamily Set\+Array\+Argument$<$N$>$(first, last)}  &Copies the elements in source range \mbox{[}{\ttfamily first}, {\ttfamily last}) to the array pointed to by the {\ttfamily N}-\/th (0-\/based) argument, which can be either a pointer or an iterator. The action does not take ownership of the elements in the source range.   \\\cline{1-2}
{\ttfamily Set\+Errno\+And\+Return(error, value)}  &Set {\ttfamily errno} to {\ttfamily error} and return {\ttfamily value}.   \\\cline{1-2}
{\ttfamily Throw(exception)}  &Throws the given exception, which can be any copyable value. Available since v1.\+1.\+0.   \\\cline{1-2}
\end{longtabu}


\subsection*{Using a Function, Functor, or Lambda as an Action}

In the following, by \char`\"{}callable\char`\"{} we mean a free function, {\ttfamily std\+::function}, functor, or lambda.

\tabulinesep=1mm
\begin{longtabu} spread 0pt [c]{*{2}{|X[-1]}|}
\hline
\rowcolor{\tableheadbgcolor}\textbf{ Action  }&\textbf{ Description   }\\\cline{1-2}
\endfirsthead
\hline
\endfoot
\hline
\rowcolor{\tableheadbgcolor}\textbf{ Action  }&\textbf{ Description   }\\\cline{1-2}
\endhead
{\ttfamily f}  &Invoke {\ttfamily f} with the arguments passed to the mock function, where {\ttfamily f} is a callable.   \\\cline{1-2}
{\ttfamily Invoke(f)}  &Invoke {\ttfamily f} with the arguments passed to the mock function, where {\ttfamily f} can be a global/static function or a functor.   \\\cline{1-2}
{\ttfamily Invoke(object\+\_\+pointer, \&class\+::method)}  &Invoke the method on the object with the arguments passed to the mock function.   \\\cline{1-2}
{\ttfamily Invoke\+Without\+Args(f)}  &Invoke {\ttfamily f}, which can be a global/static function or a functor. {\ttfamily f} must take no arguments.   \\\cline{1-2}
{\ttfamily Invoke\+Without\+Args(object\+\_\+pointer, \&class\+::method)}  &Invoke the method on the object, which takes no arguments.   \\\cline{1-2}
{\ttfamily Invoke\+Argument$<$N$>$(arg1, arg2, ..., argk)}  &Invoke the mock function\textquotesingle{}s {\ttfamily N}-\/th (0-\/based) argument, which must be a function or a functor, with the {\ttfamily k} arguments.   \\\cline{1-2}
\end{longtabu}


The return value of the invoked function is used as the return value of the action.

When defining a callable to be used with {\ttfamily Invoke$\ast$()}, you can declare any unused parameters as {\ttfamily Unused}\+:


\begin{DoxyCode}
using ::testing::Invoke;
\textcolor{keywordtype}{double} Distance(Unused, \textcolor{keywordtype}{double} x, \textcolor{keywordtype}{double} y) \{ \textcolor{keywordflow}{return} sqrt(x*x + y*y); \}
...
EXPECT\_CALL(mock, Foo(\textcolor{stringliteral}{"Hi"}, \_, \_)).WillOnce(Invoke(Distance));
\end{DoxyCode}


{\ttfamily Invoke(callback)} and {\ttfamily Invoke\+Without\+Args(callback)} take ownership of {\ttfamily callback}, which must be permanent. The type of {\ttfamily callback} must be a base callback type instead of a derived one, e.\+g.


\begin{DoxyCode}
BlockingClosure* done = \textcolor{keyword}{new} BlockingClosure;
... Invoke(done) ...;  \textcolor{comment}{// This won't compile!}

Closure* done2 = \textcolor{keyword}{new} BlockingClosure;
... Invoke(done2) ...;  \textcolor{comment}{// This works.}
\end{DoxyCode}


In {\ttfamily Invoke\+Argument$<$N$>$(...)}, if an argument needs to be passed by reference, wrap it inside {\ttfamily std\+::ref()}. For example,


\begin{DoxyCode}
using ::testing::InvokeArgument;
...
InvokeArgument<2>(5, string(\textcolor{stringliteral}{"Hi"}), std::ref(\mbox{\hyperlink{namespacefoo}{foo}}))
\end{DoxyCode}


calls the mock function\textquotesingle{}s \#2 argument, passing to it {\ttfamily 5} and {\ttfamily string(\char`\"{}\+Hi\char`\"{})} by value, and {\ttfamily foo} by reference.

\subsection*{Default Action}

\tabulinesep=1mm
\begin{longtabu} spread 0pt [c]{*{2}{|X[-1]}|}
\hline
\rowcolor{\tableheadbgcolor}\textbf{ Action  }&\textbf{ Description   }\\\cline{1-2}
\endfirsthead
\hline
\endfoot
\hline
\rowcolor{\tableheadbgcolor}\textbf{ Action  }&\textbf{ Description   }\\\cline{1-2}
\endhead
{\ttfamily Do\+Default()}  &Do the default action (specified by {\ttfamily O\+N\+\_\+\+C\+A\+L\+L()} or the built-\/in one).   \\\cline{1-2}
\end{longtabu}


\{\+: .callout .note\} {\bfseries Note\+:} due to technical reasons, {\ttfamily Do\+Default()} cannot be used inside a composite action -\/ trying to do so will result in a run-\/time error.

\subsection*{Composite Actions}

\tabulinesep=1mm
\begin{longtabu} spread 0pt [c]{*{2}{|X[-1]}|}
\hline
\rowcolor{\tableheadbgcolor}\textbf{ Action  }&\textbf{ Description   }\\\cline{1-2}
\endfirsthead
\hline
\endfoot
\hline
\rowcolor{\tableheadbgcolor}\textbf{ Action  }&\textbf{ Description   }\\\cline{1-2}
\endhead
{\ttfamily Do\+All(a1, a2, ..., an)}  &Do all actions {\ttfamily a1} to {\ttfamily an} and return the result of {\ttfamily an} in each invocation. The first {\ttfamily n -\/ 1} sub-\/actions must return void and will receive a readonly view of the arguments.   \\\cline{1-2}
{\ttfamily Ignore\+Result(a)}  &Perform action {\ttfamily a} and ignore its result. {\ttfamily a} must not return void.   \\\cline{1-2}
{\ttfamily With\+Arg$<$N$>$(a)}  &Pass the {\ttfamily N}-\/th (0-\/based) argument of the mock function to action {\ttfamily a} and perform it.   \\\cline{1-2}
{\ttfamily With\+Args$<$N1, N2, ..., Nk$>$(a)}  &Pass the selected (0-\/based) arguments of the mock function to action {\ttfamily a} and perform it.   \\\cline{1-2}
{\ttfamily Without\+Args(a)}  &Perform action {\ttfamily a} without any arguments.   \\\cline{1-2}
\end{longtabu}


\subsection*{Defining Actions}

\tabulinesep=1mm
\begin{longtabu} spread 0pt [c]{*{2}{|X[-1]}|}
\hline
\rowcolor{\tableheadbgcolor}\textbf{ Macro  }&\textbf{ Description   }\\\cline{1-2}
\endfirsthead
\hline
\endfoot
\hline
\rowcolor{\tableheadbgcolor}\textbf{ Macro  }&\textbf{ Description   }\\\cline{1-2}
\endhead
{\ttfamily A\+C\+T\+I\+O\+N(\+Sum) \{ return arg0 + arg1; \}}  &Defines an action {\ttfamily Sum()} to return the sum of the mock function\textquotesingle{}s argument \#0 and \#1.   \\\cline{1-2}
{\ttfamily A\+C\+T\+I\+O\+N\+\_\+\+P(\+Plus, n) \{ return arg0 + n; \}}  &Defines an action {\ttfamily Plus(n)} to return the sum of the mock function\textquotesingle{}s argument \#0 and {\ttfamily n}.   \\\cline{1-2}
{\ttfamily A\+C\+T\+I\+O\+N\+\_\+\+Pk(Foo, p1, ..., pk) \{ statements; \}}  &Defines a parameterized action {\ttfamily Foo(p1, ..., pk)} to execute the given {\ttfamily statements}.   \\\cline{1-2}
\end{longtabu}


The {\ttfamily A\+C\+T\+I\+O\+N$\ast$} macros cannot be used inside a function or class. 