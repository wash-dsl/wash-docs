\subsection*{Defining a \mbox{\hyperlink{classMock}{Mock}} Class}

Given


\begin{DoxyCode}
\textcolor{keyword}{class }Foo \{
 \textcolor{keyword}{public}:
  \textcolor{keyword}{virtual} ~Foo();
  \textcolor{keyword}{virtual} \textcolor{keywordtype}{int} GetSize() \textcolor{keyword}{const} = 0;
  \textcolor{keyword}{virtual} \textcolor{keywordtype}{string} Describe(\textcolor{keyword}{const} \textcolor{keywordtype}{char}* name) = 0;
  \textcolor{keyword}{virtual} \textcolor{keywordtype}{string} Describe(\textcolor{keywordtype}{int} type) = 0;
  \textcolor{keyword}{virtual} \textcolor{keywordtype}{bool} Process(Bar elem, \textcolor{keywordtype}{int} count) = 0;
\};
\end{DoxyCode}


(note that {\ttfamily $\sim$\+Foo()} {\bfseries must} be virtual) we can define its mock as


\begin{DoxyCode}
\textcolor{preprocessor}{#include <gmock/gmock.h>}

\textcolor{keyword}{class }MockFoo : \textcolor{keyword}{public} Foo \{
 \textcolor{keyword}{public}:
  MOCK\_METHOD(\textcolor{keywordtype}{int}, GetSize, (), (\textcolor{keyword}{const}, \textcolor{keyword}{override}));
  MOCK\_METHOD(\textcolor{keywordtype}{string}, Describe, (\textcolor{keyword}{const} \textcolor{keywordtype}{char}* name), (\textcolor{keyword}{override}));
  MOCK\_METHOD(\textcolor{keywordtype}{string}, Describe, (\textcolor{keywordtype}{int} type), (\textcolor{keyword}{override}));
  MOCK\_METHOD(\textcolor{keywordtype}{bool}, Process, (Bar elem, \textcolor{keywordtype}{int} count), (\textcolor{keyword}{override}));
\};
\end{DoxyCode}


To create a \char`\"{}nice\char`\"{} mock, which ignores all uninteresting calls, a \char`\"{}naggy\char`\"{} mock, which warns on all uninteresting calls, or a \char`\"{}strict\char`\"{} mock, which treats them as failures\+:


\begin{DoxyCode}
using ::testing::NiceMock;
using ::testing::NaggyMock;
using ::testing::StrictMock;

NiceMock<MockFoo> nice\_foo;      \textcolor{comment}{// The type is a subclass of MockFoo.}
NaggyMock<MockFoo> naggy\_foo;    \textcolor{comment}{// The type is a subclass of MockFoo.}
StrictMock<MockFoo> strict\_foo;  \textcolor{comment}{// The type is a subclass of MockFoo.}
\end{DoxyCode}


\{\+: .callout .note\} {\bfseries Note\+:} A mock object is currently naggy by default. We may make it nice by default in the future.

Class templates can be mocked just like any class.

To mock


\begin{DoxyCode}
\textcolor{keyword}{template} <\textcolor{keyword}{typename} Elem>
\textcolor{keyword}{class }StackInterface \{
 \textcolor{keyword}{public}:
  \textcolor{keyword}{virtual} ~StackInterface();
  \textcolor{keyword}{virtual} \textcolor{keywordtype}{int} GetSize() \textcolor{keyword}{const} = 0;
  \textcolor{keyword}{virtual} \textcolor{keywordtype}{void} Push(\textcolor{keyword}{const} Elem& x) = 0;
\};
\end{DoxyCode}


(note that all member functions that are mocked, including {\ttfamily $\sim$\+Stack\+Interface()} {\bfseries must} be virtual).


\begin{DoxyCode}
\textcolor{keyword}{template} <\textcolor{keyword}{typename} Elem>
\textcolor{keyword}{class }MockStack : \textcolor{keyword}{public} StackInterface<Elem> \{
 \textcolor{keyword}{public}:
  MOCK\_METHOD(\textcolor{keywordtype}{int}, GetSize, (), (\textcolor{keyword}{const}, \textcolor{keyword}{override}));
  MOCK\_METHOD(\textcolor{keywordtype}{void}, Push, (\textcolor{keyword}{const} Elem& x), (\textcolor{keyword}{override}));
\};
\end{DoxyCode}


\subsubsection*{Specifying Calling Conventions for \mbox{\hyperlink{classMock}{Mock}} Functions}

If your mock function doesn\textquotesingle{}t use the default calling convention, you can specify it by adding {\ttfamily Calltype(convention)} to {\ttfamily M\+O\+C\+K\+\_\+\+M\+E\+T\+H\+OD}\textquotesingle{}s 4th parameter. For example,


\begin{DoxyCode}
MOCK\_METHOD(\textcolor{keywordtype}{bool}, Foo, (\textcolor{keywordtype}{int} n), (Calltype(STDMETHODCALLTYPE)));
MOCK\_METHOD(\textcolor{keywordtype}{int}, Bar, (\textcolor{keywordtype}{double} x, \textcolor{keywordtype}{double} y),
            (\textcolor{keyword}{const}, Calltype(STDMETHODCALLTYPE)));
\end{DoxyCode}


where {\ttfamily S\+T\+D\+M\+E\+T\+H\+O\+D\+C\+A\+L\+L\+T\+Y\+PE} is defined by {\ttfamily $<$objbase.\+h$>$} on Windows.

The typical work flow is\+:


\begin{DoxyEnumerate}
\item Import the g\+Mock names you need to use. All g\+Mock symbols are in the {\ttfamily testing} namespace unless they are macros or otherwise noted.
\item Create the mock objects.
\item Optionally, set the default actions of the mock objects.
\item Set your expectations on the mock objects (How will they be called? What will they do?).
\item Exercise code that uses the mock objects; if necessary, check the result using googletest assertions.
\item When a mock object is destructed, g\+Mock automatically verifies that all expectations on it have been satisfied.
\end{DoxyEnumerate}

Here\textquotesingle{}s an example\+:


\begin{DoxyCode}
using ::testing::Return;                          \textcolor{comment}{// #1}

TEST(BarTest, DoesThis) \{
  MockFoo \mbox{\hyperlink{namespacefoo}{foo}};                                    \textcolor{comment}{// #2}

  ON\_CALL(\mbox{\hyperlink{namespacefoo}{foo}}, GetSize())                         \textcolor{comment}{// #3}
      .WillByDefault(Return(1));
  \textcolor{comment}{// ... other default actions ...}

  EXPECT\_CALL(\mbox{\hyperlink{namespacefoo}{foo}}, Describe(5))                   \textcolor{comment}{// #4}
      .Times(3)
      .WillRepeatedly(Return(\textcolor{stringliteral}{"Category 5"}));
  \textcolor{comment}{// ... other expectations ...}

  EXPECT\_EQ(MyProductionFunction(&\mbox{\hyperlink{namespacefoo}{foo}}), \textcolor{stringliteral}{"good"});  \textcolor{comment}{// #5}
\}                                                 \textcolor{comment}{// #6}
\end{DoxyCode}


g\+Mock has a {\bfseries built-\/in default action} for any function that returns {\ttfamily void}, {\ttfamily bool}, a numeric value, or a pointer. In C++11, it will additionally returns the default-\/constructed value, if one exists for the given type.

To customize the default action for functions with return type {\ttfamily T}, use \href{reference/mocking.md#DefaultValue}{\tt {\ttfamily Default\+Value$<$T$>$}}. For example\+:


\begin{DoxyCode}
\textcolor{comment}{// Sets the default action for return type std::unique\_ptr<Buzz> to}
\textcolor{comment}{// creating a new Buzz every time.}
DefaultValue<std::unique\_ptr<Buzz>>::SetFactory(
    [] \{ \textcolor{keywordflow}{return} std::make\_unique<Buzz>(AccessLevel::kInternal); \});

\textcolor{comment}{// When this fires, the default action of MakeBuzz() will run, which}
\textcolor{comment}{// will return a new Buzz object.}
EXPECT\_CALL(mock\_buzzer\_, MakeBuzz(\textcolor{stringliteral}{"hello"})).Times(AnyNumber());

\textcolor{keyword}{auto} buzz1 = mock\_buzzer\_.MakeBuzz(\textcolor{stringliteral}{"hello"});
\textcolor{keyword}{auto} buzz2 = mock\_buzzer\_.MakeBuzz(\textcolor{stringliteral}{"hello"});
EXPECT\_NE(buzz1, \textcolor{keyword}{nullptr});
EXPECT\_NE(buzz2, \textcolor{keyword}{nullptr});
EXPECT\_NE(buzz1, buzz2);

\textcolor{comment}{// Resets the default action for return type std::unique\_ptr<Buzz>,}
\textcolor{comment}{// to avoid interfere with other tests.}
DefaultValue<std::unique\_ptr<Buzz>>::Clear();
\end{DoxyCode}


To customize the default action for a particular method of a specific mock object, use \href{reference/mocking.md#ON_CALL}{\tt {\ttfamily O\+N\+\_\+\+C\+A\+LL}}. {\ttfamily O\+N\+\_\+\+C\+A\+LL} has a similar syntax to {\ttfamily E\+X\+P\+E\+C\+T\+\_\+\+C\+A\+LL}, but it is used for setting default behaviors when you do not require that the mock method is called. See \href{gmock_cook_book.md#UseOnCall}{\tt Knowing When to Expect} for a more detailed discussion.

See \href{reference/mocking.md#EXPECT_CALL}{\tt {\ttfamily E\+X\+P\+E\+C\+T\+\_\+\+C\+A\+LL}} in the Mocking Reference.

See the Matchers Reference.

See the Actions Reference.

See the \href{reference/mocking.md#EXPECT_CALL.Times}{\tt {\ttfamily Times} clause} of {\ttfamily E\+X\+P\+E\+C\+T\+\_\+\+C\+A\+LL} in the Mocking Reference.

\subsection*{Expectation Order}

By default, expectations can be matched in {\itshape any} order. If some or all expectations must be matched in a given order, you can use the \href{reference/mocking.md#EXPECT_CALL.After}{\tt {\ttfamily After} clause} or \href{reference/mocking.md#EXPECT_CALL.InSequence}{\tt {\ttfamily In\+Sequence} clause} of {\ttfamily E\+X\+P\+E\+C\+T\+\_\+\+C\+A\+LL}, or use an \href{reference/mocking.md#InSequence}{\tt {\ttfamily In\+Sequence} object}.

\subsection*{Verifying and Resetting a \mbox{\hyperlink{classMock}{Mock}}}

g\+Mock will verify the expectations on a mock object when it is destructed, or you can do it earlier\+:


\begin{DoxyCode}
using ::testing::Mock;
...
\textcolor{comment}{// Verifies and removes the expectations on mock\_obj;}
\textcolor{comment}{// returns true if and only if successful.}
Mock::VerifyAndClearExpectations(&mock\_obj);
...
\textcolor{comment}{// Verifies and removes the expectations on mock\_obj;}
\textcolor{comment}{// also removes the default actions set by ON\_CALL();}
\textcolor{comment}{// returns true if and only if successful.}
Mock::VerifyAndClear(&mock\_obj);
\end{DoxyCode}


Do not set new expectations after verifying and clearing a mock after its use. Setting expectations after code that exercises the mock has undefined behavior. See \href{gmock_for_dummies.md#using-mocks-in-tests}{\tt Using Mocks in Tests} for more information.

You can also tell g\+Mock that a mock object can be leaked and doesn\textquotesingle{}t need to be verified\+:


\begin{DoxyCode}
Mock::AllowLeak(&mock\_obj);
\end{DoxyCode}


\subsection*{\mbox{\hyperlink{classMock}{Mock}} Classes}

g\+Mock defines a convenient mock class template


\begin{DoxyCode}
\textcolor{keyword}{class }MockFunction<R(A1, ..., An)> \{
 \textcolor{keyword}{public}:
  MOCK\_METHOD(R, Call, (A1, ..., An));
\};
\end{DoxyCode}


See this \href{gmock_cook_book.md#UsingCheckPoints}{\tt recipe} for one application of it.

\subsection*{Flags}

\tabulinesep=1mm
\begin{longtabu} spread 0pt [c]{*{2}{|X[-1]}|}
\hline
\rowcolor{\tableheadbgcolor}\textbf{ Flag  }&\textbf{ Description   }\\\cline{1-2}
\endfirsthead
\hline
\endfoot
\hline
\rowcolor{\tableheadbgcolor}\textbf{ Flag  }&\textbf{ Description   }\\\cline{1-2}
\endhead
{\ttfamily -\/-\/gmock\+\_\+catch\+\_\+leaked\+\_\+mocks=0}  &Don\textquotesingle{}t report leaked mock objects as failures.   \\\cline{1-2}
{\ttfamily -\/-\/gmock\+\_\+verbose=L\+E\+V\+EL}  &Sets the default verbosity level ({\ttfamily info}, {\ttfamily warning}, or {\ttfamily error}) of Google \mbox{\hyperlink{classMock}{Mock}} messages.   \\\cline{1-2}
\end{longtabu}
