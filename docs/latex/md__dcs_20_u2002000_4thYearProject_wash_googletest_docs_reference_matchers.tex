A {\bfseries matcher} matches a {\itshape single} argument. You can use it inside {\ttfamily O\+N\+\_\+\+C\+A\+L\+L()} or {\ttfamily E\+X\+P\+E\+C\+T\+\_\+\+C\+A\+L\+L()}, or use it to validate a value directly using two macros\+:

\tabulinesep=1mm
\begin{longtabu} spread 0pt [c]{*{2}{|X[-1]}|}
\hline
\rowcolor{\tableheadbgcolor}\textbf{ Macro  }&\textbf{ Description   }\\\cline{1-2}
\endfirsthead
\hline
\endfoot
\hline
\rowcolor{\tableheadbgcolor}\textbf{ Macro  }&\textbf{ Description   }\\\cline{1-2}
\endhead
{\ttfamily E\+X\+P\+E\+C\+T\+\_\+\+T\+H\+A\+T(actual\+\_\+value, matcher)}  &Asserts that {\ttfamily actual\+\_\+value} matches {\ttfamily matcher}.   \\\cline{1-2}
{\ttfamily A\+S\+S\+E\+R\+T\+\_\+\+T\+H\+A\+T(actual\+\_\+value, matcher)}  &The same as {\ttfamily E\+X\+P\+E\+C\+T\+\_\+\+T\+H\+A\+T(actual\+\_\+value, matcher)}, except that it generates a {\bfseries fatal} failure.   \\\cline{1-2}
\end{longtabu}


\{\+: .callout .warning\} {\bfseries W\+A\+R\+N\+I\+NG\+:} Equality matching via {\ttfamily E\+X\+P\+E\+C\+T\+\_\+\+T\+H\+A\+T(actual\+\_\+value, expected\+\_\+value)} is supported, however note that implicit conversions can cause surprising results. For example, {\ttfamily E\+X\+P\+E\+C\+T\+\_\+\+T\+H\+AT(some\+\_\+bool, \char`\"{}some string\char`\"{})} will compile and may pass unintentionally.

{\bfseries B\+E\+ST P\+R\+A\+C\+T\+I\+CE\+:} Prefer to make the comparison explicit via {\ttfamily E\+X\+P\+E\+C\+T\+\_\+\+T\+H\+A\+T(actual\+\_\+value, Eq(expected\+\_\+value))} or {\ttfamily E\+X\+P\+E\+C\+T\+\_\+\+EQ(actual\+\_\+value, expected\+\_\+value)}.

Built-\/in matchers (where {\ttfamily argument} is the function argument, e.\+g. {\ttfamily actual\+\_\+value} in the example above, or when used in the context of {\ttfamily E\+X\+P\+E\+C\+T\+\_\+\+C\+A\+L\+L(mock\+\_\+object, method(matchers))}, the arguments of {\ttfamily method}) are divided into several categories. All matchers are defined in the {\ttfamily \+::testing} namespace unless otherwise noted.

\subsection*{Wildcard}

\tabulinesep=1mm
\begin{longtabu} spread 0pt [c]{*{2}{|X[-1]}|}
\hline
\rowcolor{\tableheadbgcolor}\textbf{ Matcher  }&\textbf{ Description \+:-\/-\/-\/-\/-\/-\/-\/-\/-\/-\/-\/-\/-\/-\/-\/-\/-\/-\/-\/-\/-\/-\/-\/---   }\\\cline{1-2}
\endfirsthead
\hline
\endfoot
\hline
\rowcolor{\tableheadbgcolor}\textbf{ Matcher  }&\textbf{ Description \+:-\/-\/-\/-\/-\/-\/-\/-\/-\/-\/-\/-\/-\/-\/-\/-\/-\/-\/-\/-\/-\/-\/-\/---   }\\\cline{1-2}
\endhead
{\ttfamily \+\_\+}  &{\ttfamily argument} can be any value of the correct type.   \\\cline{1-2}
{\ttfamily A$<$type$>$()} or {\ttfamily An$<$type$>$()}  &{\ttfamily argument} can be any value of type {\ttfamily type}.   \\\cline{1-2}
\end{longtabu}


\subsection*{Generic Comparison}

\tabulinesep=1mm
\begin{longtabu} spread 0pt [c]{*{2}{|X[-1]}|}
\hline
\rowcolor{\tableheadbgcolor}\textbf{ Matcher  }&\textbf{ Description   }\\\cline{1-2}
\endfirsthead
\hline
\endfoot
\hline
\rowcolor{\tableheadbgcolor}\textbf{ Matcher  }&\textbf{ Description   }\\\cline{1-2}
\endhead
{\ttfamily Eq(value)} or {\ttfamily value}  &{\ttfamily argument == value}   \\\cline{1-2}
{\ttfamily Ge(value)}  &{\ttfamily argument $>$= value}   \\\cline{1-2}
{\ttfamily Gt(value)}  &{\ttfamily argument $>$ value}   \\\cline{1-2}
{\ttfamily Le(value)}  &{\ttfamily argument $<$= value}   \\\cline{1-2}
{\ttfamily Lt(value)}  &{\ttfamily argument $<$ value}   \\\cline{1-2}
{\ttfamily Ne(value)}  &{\ttfamily argument != value}   \\\cline{1-2}
{\ttfamily Is\+False()}  &{\ttfamily argument} evaluates to {\ttfamily false} in a Boolean context.   \\\cline{1-2}
{\ttfamily Is\+True()}  &{\ttfamily argument} evaluates to {\ttfamily true} in a Boolean context.   \\\cline{1-2}
{\ttfamily Is\+Null()}  &{\ttfamily argument} is a {\ttfamily N\+U\+LL} pointer (raw or smart).   \\\cline{1-2}
{\ttfamily Not\+Null()}  &{\ttfamily argument} is a non-\/null pointer (raw or smart).   \\\cline{1-2}
{\ttfamily Optional(m)}  &{\ttfamily argument} is {\ttfamily optional$<$$>$} that contains a value matching {\ttfamily m}. (For testing whether an {\ttfamily optional$<$$>$} is set, check for equality with {\ttfamily nullopt}. You may need to use {\ttfamily Eq(nullopt)} if the inner type doesn\textquotesingle{}t have {\ttfamily ==}.)   \\\cline{1-2}
{\ttfamily Variant\+With$<$T$>$(m)}  &{\ttfamily argument} is {\ttfamily variant$<$$>$} that holds the alternative of type T with a value matching {\ttfamily m}.   \\\cline{1-2}
{\ttfamily Ref(variable)}  &{\ttfamily argument} is a reference to {\ttfamily variable}.   \\\cline{1-2}
{\ttfamily Typed\+Eq$<$type$>$(value)}  &{\ttfamily argument} has type {\ttfamily type} and is equal to {\ttfamily value}. You may need to use this instead of {\ttfamily Eq(value)} when the mock function is overloaded.   \\\cline{1-2}
\end{longtabu}


Except {\ttfamily Ref()}, these matchers make a {\itshape copy} of {\ttfamily value} in case it\textquotesingle{}s modified or destructed later. If the compiler complains that {\ttfamily value} doesn\textquotesingle{}t have a public copy constructor, try wrap it in {\ttfamily std\+::ref()}, e.\+g. {\ttfamily Eq(std\+::ref(non\+\_\+copyable\+\_\+value))}. If you do that, make sure {\ttfamily non\+\_\+copyable\+\_\+value} is not changed afterwards, or the meaning of your matcher will be changed.

{\ttfamily Is\+True} and {\ttfamily Is\+False} are useful when you need to use a matcher, or for types that can be explicitly converted to Boolean, but are not implicitly converted to Boolean. In other cases, you can use the basic \href{assertions.md#boolean}{\tt {\ttfamily E\+X\+P\+E\+C\+T\+\_\+\+T\+R\+UE} and {\ttfamily E\+X\+P\+E\+C\+T\+\_\+\+F\+A\+L\+SE}} assertions.

\tabulinesep=1mm
\begin{longtabu} spread 0pt [c]{*{2}{|X[-1]}|}
\hline
\rowcolor{\tableheadbgcolor}\textbf{ Matcher  }&\textbf{ Description   }\\\cline{1-2}
\endfirsthead
\hline
\endfoot
\hline
\rowcolor{\tableheadbgcolor}\textbf{ Matcher  }&\textbf{ Description   }\\\cline{1-2}
\endhead
{\ttfamily Double\+Eq(a\+\_\+double)}  &{\ttfamily argument} is a {\ttfamily double} value approximately equal to {\ttfamily a\+\_\+double}, treating two Na\+Ns as unequal.   \\\cline{1-2}
{\ttfamily Float\+Eq(a\+\_\+float)}  &{\ttfamily argument} is a {\ttfamily float} value approximately equal to {\ttfamily a\+\_\+float}, treating two Na\+Ns as unequal.   \\\cline{1-2}
{\ttfamily Nan\+Sensitive\+Double\+Eq(a\+\_\+double)}  &{\ttfamily argument} is a {\ttfamily double} value approximately equal to {\ttfamily a\+\_\+double}, treating two Na\+Ns as equal.   \\\cline{1-2}
{\ttfamily Nan\+Sensitive\+Float\+Eq(a\+\_\+float)}  &{\ttfamily argument} is a {\ttfamily float} value approximately equal to {\ttfamily a\+\_\+float}, treating two Na\+Ns as equal.   \\\cline{1-2}
{\ttfamily Is\+Nan()}  &{\ttfamily argument} is any floating-\/point type with a NaN value.   \\\cline{1-2}
\end{longtabu}


The above matchers use U\+L\+P-\/based comparison (the same as used in googletest). They automatically pick a reasonable error bound based on the absolute value of the expected value. {\ttfamily Double\+Eq()} and {\ttfamily Float\+Eq()} conform to the I\+E\+EE standard, which requires comparing two Na\+Ns for equality to return false. The {\ttfamily Nan\+Sensitive$\ast$} version instead treats two Na\+Ns as equal, which is often what a user wants.

\tabulinesep=1mm
\begin{longtabu} spread 0pt [c]{*{2}{|X[-1]}|}
\hline
\rowcolor{\tableheadbgcolor}\textbf{ Matcher  }&\textbf{ Description   }\\\cline{1-2}
\endfirsthead
\hline
\endfoot
\hline
\rowcolor{\tableheadbgcolor}\textbf{ Matcher  }&\textbf{ Description   }\\\cline{1-2}
\endhead
{\ttfamily Double\+Near(a\+\_\+double, max\+\_\+abs\+\_\+error)}  &{\ttfamily argument} is a {\ttfamily double} value close to {\ttfamily a\+\_\+double} (absolute error $<$= {\ttfamily max\+\_\+abs\+\_\+error}), treating two Na\+Ns as unequal.   \\\cline{1-2}
{\ttfamily Float\+Near(a\+\_\+float, max\+\_\+abs\+\_\+error)}  &{\ttfamily argument} is a {\ttfamily float} value close to {\ttfamily a\+\_\+float} (absolute error $<$= {\ttfamily max\+\_\+abs\+\_\+error}), treating two Na\+Ns as unequal.   \\\cline{1-2}
{\ttfamily Nan\+Sensitive\+Double\+Near(a\+\_\+double, max\+\_\+abs\+\_\+error)}  &{\ttfamily argument} is a {\ttfamily double} value close to {\ttfamily a\+\_\+double} (absolute error $<$= {\ttfamily max\+\_\+abs\+\_\+error}), treating two Na\+Ns as equal.   \\\cline{1-2}
{\ttfamily Nan\+Sensitive\+Float\+Near(a\+\_\+float, max\+\_\+abs\+\_\+error)}  &{\ttfamily argument} is a {\ttfamily float} value close to {\ttfamily a\+\_\+float} (absolute error $<$= {\ttfamily max\+\_\+abs\+\_\+error}), treating two Na\+Ns as equal.   \\\cline{1-2}
\end{longtabu}


\subsection*{String Matchers}

The {\ttfamily argument} can be either a C string or a C++ string object\+:

\tabulinesep=1mm
\begin{longtabu} spread 0pt [c]{*{2}{|X[-1]}|}
\hline
\rowcolor{\tableheadbgcolor}\textbf{ Matcher  }&\textbf{ Description   }\\\cline{1-2}
\endfirsthead
\hline
\endfoot
\hline
\rowcolor{\tableheadbgcolor}\textbf{ Matcher  }&\textbf{ Description   }\\\cline{1-2}
\endhead
{\ttfamily Contains\+Regex(string)}  &{\ttfamily argument} matches the given regular expression.   \\\cline{1-2}
{\ttfamily Ends\+With(suffix)}  &{\ttfamily argument} ends with string {\ttfamily suffix}.   \\\cline{1-2}
{\ttfamily Has\+Substr(string)}  &{\ttfamily argument} contains {\ttfamily string} as a sub-\/string.   \\\cline{1-2}
{\ttfamily Is\+Empty()}  &{\ttfamily argument} is an empty string.   \\\cline{1-2}
{\ttfamily Matches\+Regex(string)}  &{\ttfamily argument} matches the given regular expression with the match starting at the first character and ending at the last character.   \\\cline{1-2}
{\ttfamily Starts\+With(prefix)}  &{\ttfamily argument} starts with string {\ttfamily prefix}.   \\\cline{1-2}
{\ttfamily Str\+Case\+Eq(string)}  &{\ttfamily argument} is equal to {\ttfamily string}, ignoring case.   \\\cline{1-2}
{\ttfamily Str\+Case\+Ne(string)}  &{\ttfamily argument} is not equal to {\ttfamily string}, ignoring case.   \\\cline{1-2}
{\ttfamily Str\+Eq(string)}  &{\ttfamily argument} is equal to {\ttfamily string}.   \\\cline{1-2}
{\ttfamily Str\+Ne(string)}  &{\ttfamily argument} is not equal to {\ttfamily string}.   \\\cline{1-2}
{\ttfamily When\+Base64\+Unescaped(m)}  &{\ttfamily argument} is a base-\/64 escaped string whose unescaped string matches {\ttfamily m}. The web-\/safe format from \href{https://www.rfc-editor.org/rfc/rfc4648#section-5}{\tt R\+FC 4648} is supported.   \\\cline{1-2}
\end{longtabu}


{\ttfamily Contains\+Regex()} and {\ttfamily Matches\+Regex()} take ownership of the {\ttfamily RE} object. They use the regular expression syntax defined \href{../advanced.md#regular-expression-syntax}{\tt here}. All of these matchers, except {\ttfamily Contains\+Regex()} and {\ttfamily Matches\+Regex()} work for wide strings as well.

\subsection*{Container Matchers}

Most S\+T\+L-\/style containers support {\ttfamily ==}, so you can use {\ttfamily Eq(expected\+\_\+container)} or simply {\ttfamily expected\+\_\+container} to match a container exactly. If you want to write the elements in-\/line, match them more flexibly, or get more informative messages, you can use\+:

\tabulinesep=1mm
\begin{longtabu} spread 0pt [c]{*{2}{|X[-1]}|}
\hline
\rowcolor{\tableheadbgcolor}\textbf{ Matcher  }&\textbf{ Description   }\\\cline{1-2}
\endfirsthead
\hline
\endfoot
\hline
\rowcolor{\tableheadbgcolor}\textbf{ Matcher  }&\textbf{ Description   }\\\cline{1-2}
\endhead
{\ttfamily Begin\+End\+Distance\+Is(m)}  &{\ttfamily argument} is a container whose {\ttfamily begin()} and {\ttfamily end()} iterators are separated by a number of increments matching {\ttfamily m}. E.\+g. {\ttfamily Begin\+End\+Distance\+Is(2)} or {\ttfamily Begin\+End\+Distance\+Is(\+Lt(2))}. For containers that define a {\ttfamily size()} method, {\ttfamily Size\+Is(m)} may be more efficient.   \\\cline{1-2}
{\ttfamily Container\+Eq(container)}  &The same as {\ttfamily Eq(container)} except that the failure message also includes which elements are in one container but not the other.   \\\cline{1-2}
{\ttfamily Contains(e)}  &{\ttfamily argument} contains an element that matches {\ttfamily e}, which can be either a value or a matcher.   \\\cline{1-2}
{\ttfamily Contains(e).Times(n)}  &{\ttfamily argument} contains elements that match {\ttfamily e}, which can be either a value or a matcher, and the number of matches is {\ttfamily n}, which can be either a value or a matcher. Unlike the plain {\ttfamily Contains} and {\ttfamily Each} this allows to check for arbitrary occurrences including testing for absence with {\ttfamily Contains(e).Times(0)}.   \\\cline{1-2}
{\ttfamily Each(e)}  &{\ttfamily argument} is a container where {\itshape every} element matches {\ttfamily e}, which can be either a value or a matcher.   \\\cline{1-2}
{\ttfamily Elements\+Are(e0, e1, ..., en)}  &{\ttfamily argument} has {\ttfamily n + 1} elements, where the {\itshape i}-\/th element matches {\ttfamily ei}, which can be a value or a matcher.   \\\cline{1-2}
{\ttfamily Elements\+Are\+Array(\{e0, e1, ..., en\})}, {\ttfamily Elements\+Are\+Array(a\+\_\+container)}, {\ttfamily Elements\+Are\+Array(begin, end)}, {\ttfamily Elements\+Are\+Array(array)}, or {\ttfamily Elements\+Are\+Array(array, count)}  &The same as {\ttfamily Elements\+Are()} except that the expected element values/matchers come from an initializer list, S\+T\+L-\/style container, iterator range, or C-\/style array.   \\\cline{1-2}
{\ttfamily Is\+Empty()}  &{\ttfamily argument} is an empty container ({\ttfamily container.\+empty()}).   \\\cline{1-2}
{\ttfamily Is\+Subset\+Of(\{e0, e1, ..., en\})}, {\ttfamily Is\+Subset\+Of(a\+\_\+container)}, {\ttfamily Is\+Subset\+Of(begin, end)}, {\ttfamily Is\+Subset\+Of(array)}, or {\ttfamily Is\+Subset\+Of(array, count)}  &{\ttfamily argument} matches {\ttfamily Unordered\+Elements\+Are(x0, x1, ..., xk)} for some subset {\ttfamily \{x0, x1, ..., xk\}} of the expected matchers.   \\\cline{1-2}
{\ttfamily Is\+Superset\+Of(\{e0, e1, ..., en\})}, {\ttfamily Is\+Superset\+Of(a\+\_\+container)}, {\ttfamily Is\+Superset\+Of(begin, end)}, {\ttfamily Is\+Superset\+Of(array)}, or {\ttfamily Is\+Superset\+Of(array, count)}  &Some subset of {\ttfamily argument} matches {\ttfamily Unordered\+Elements\+Are(}expected matchers{\ttfamily )}.   \\\cline{1-2}
{\ttfamily Pointwise(m, container)}, {\ttfamily Pointwise(m, \{e0, e1, ..., en\})}  &{\ttfamily argument} contains the same number of elements as in {\ttfamily container}, and for all i, (the i-\/th element in {\ttfamily argument}, the i-\/th element in {\ttfamily container}) match {\ttfamily m}, which is a matcher on 2-\/tuples. E.\+g. {\ttfamily Pointwise(\+Le(), upper\+\_\+bounds)} verifies that each element in {\ttfamily argument} doesn\textquotesingle{}t exceed the corresponding element in {\ttfamily upper\+\_\+bounds}. See more detail below.   \\\cline{1-2}
{\ttfamily Size\+Is(m)}  &{\ttfamily argument} is a container whose size matches {\ttfamily m}. E.\+g. {\ttfamily Size\+Is(2)} or {\ttfamily Size\+Is(\+Lt(2))}.   \\\cline{1-2}
{\ttfamily Unordered\+Elements\+Are(e0, e1, ..., en)}  &{\ttfamily argument} has {\ttfamily n + 1} elements, and under {\itshape some} permutation of the elements, each element matches an {\ttfamily ei} (for a different {\ttfamily i}), which can be a value or a matcher.   \\\cline{1-2}
{\ttfamily Unordered\+Elements\+Are\+Array(\{e0, e1, ..., en\})}, {\ttfamily Unordered\+Elements\+Are\+Array(a\+\_\+container)}, {\ttfamily Unordered\+Elements\+Are\+Array(begin, end)}, {\ttfamily Unordered\+Elements\+Are\+Array(array)}, or {\ttfamily Unordered\+Elements\+Are\+Array(array, count)}  &The same as {\ttfamily Unordered\+Elements\+Are()} except that the expected element values/matchers come from an initializer list, S\+T\+L-\/style container, iterator range, or C-\/style array.   \\\cline{1-2}
{\ttfamily Unordered\+Pointwise(m, container)}, {\ttfamily Unordered\+Pointwise(m, \{e0, e1, ..., en\})}  &Like {\ttfamily Pointwise(m, container)}, but ignores the order of elements.   \\\cline{1-2}
{\ttfamily When\+Sorted(m)}  &When {\ttfamily argument} is sorted using the {\ttfamily $<$} operator, it matches container matcher {\ttfamily m}. E.\+g. {\ttfamily When\+Sorted(\+Elements\+Are(1, 2, 3))} verifies that {\ttfamily argument} contains elements 1, 2, and 3, ignoring order.   \\\cline{1-2}
{\ttfamily When\+Sorted\+By(comparator, m)}  &The same as {\ttfamily When\+Sorted(m)}, except that the given comparator instead of {\ttfamily $<$} is used to sort {\ttfamily argument}. E.\+g. {\ttfamily When\+Sorted\+By(std\+::greater(), Elements\+Are(3, 2, 1))}.   \\\cline{1-2}
\end{longtabu}


{\bfseries Notes\+:}


\begin{DoxyItemize}
\item These matchers can also match\+:
\begin{DoxyEnumerate}
\item a native array passed by reference (e.\+g. in {\ttfamily Foo(const int (\&a)\mbox{[}5\mbox{]})}), and
\item an array passed as a pointer and a count (e.\+g. in {\ttfamily Bar(const T$\ast$ buffer, int len)} -- see \href{#MultiArgMatchers}{\tt Multi-\/argument Matchers}).
\end{DoxyEnumerate}
\item The array being matched may be multi-\/dimensional (i.\+e. its elements can be arrays).
\item {\ttfamily m} in {\ttfamily Pointwise(m, ...)} and {\ttfamily Unordered\+Pointwise(m, ...)} should be a matcher for {\ttfamily \+::std\+::tuple$<$T, U$>$} where {\ttfamily T} and {\ttfamily U} are the element type of the actual container and the expected container, respectively. For example, to compare two {\ttfamily Foo} containers where {\ttfamily Foo} doesn\textquotesingle{}t support {\ttfamily operator==}, one might write\+:

\`{}\`{}\`{}cpp M\+A\+T\+C\+H\+ER(Foo\+Eq, \char`\"{}\char`\"{}) \{ return std\+::get$<$0$>$(arg).Equals(std\+::get$<$1$>$(arg)); \} ... E\+X\+P\+E\+C\+T\+\_\+\+T\+H\+AT(actual\+\_\+foos, Pointwise(\+Foo\+Eq(), expected\+\_\+foos)); \`{}\`{}\`{}
\end{DoxyItemize}

\subsection*{Member Matchers}

\tabulinesep=1mm
\begin{longtabu} spread 0pt [c]{*{2}{|X[-1]}|}
\hline
\rowcolor{\tableheadbgcolor}\textbf{ Matcher  }&\textbf{ Description   }\\\cline{1-2}
\endfirsthead
\hline
\endfoot
\hline
\rowcolor{\tableheadbgcolor}\textbf{ Matcher  }&\textbf{ Description   }\\\cline{1-2}
\endhead
{\ttfamily Field(\&class\+::field, m)}  &{\ttfamily argument.\+field} (or {\ttfamily argument-\/$>$field} when {\ttfamily argument} is a plain pointer) matches matcher {\ttfamily m}, where {\ttfamily argument} is an object of type {\itshape class}.   \\\cline{1-2}
{\ttfamily Field(field\+\_\+name, \&class\+::field, m)}  &The same as the two-\/parameter version, but provides a better error message.   \\\cline{1-2}
{\ttfamily Key(e)}  &{\ttfamily argument.\+first} matches {\ttfamily e}, which can be either a value or a matcher. E.\+g. {\ttfamily Contains(Key(\+Le(5)))} can verify that a {\ttfamily map} contains a key {\ttfamily $<$= 5}.   \\\cline{1-2}
{\ttfamily Pair(m1, m2)}  &{\ttfamily argument} is an {\ttfamily std\+::pair} whose {\ttfamily first} field matches {\ttfamily m1} and {\ttfamily second} field matches {\ttfamily m2}.   \\\cline{1-2}
{\ttfamily Fields\+Are(m...)}  &{\ttfamily argument} is a compatible object where each field matches piecewise with the matchers {\ttfamily m...}. A compatible object is any that supports the {\ttfamily std\+::tuple\+\_\+size$<$Obj$>$}+{\ttfamily get$<$I$>$(obj)} protocol. In C++17 and up this also supports types compatible with structured bindings, like aggregates.   \\\cline{1-2}
{\ttfamily Property(\&class\+::property, m)}  &{\ttfamily argument.\+property()} (or {\ttfamily argument-\/$>$property()} when {\ttfamily argument} is a plain pointer) matches matcher {\ttfamily m}, where {\ttfamily argument} is an object of type {\itshape class}. The method {\ttfamily property()} must take no argument and be declared as {\ttfamily const}.   \\\cline{1-2}
{\ttfamily Property(property\+\_\+name, \&class\+::property, m)}  &The same as the two-\/parameter version, but provides a better error message.   \\\cline{1-2}
\end{longtabu}


{\bfseries Notes\+:}


\begin{DoxyItemize}
\item You can use {\ttfamily Fields\+Are()} to match any type that supports structured bindings, such as {\ttfamily std\+::tuple}, {\ttfamily std\+::pair}, {\ttfamily std\+::array}, and aggregate types. For example\+:

\`{}\`{}\`{}cpp std\+::tuple$<$int, std\+::string$>$ my\+\_\+tuple\{7, \char`\"{}hello world\char`\"{}\}; E\+X\+P\+E\+C\+T\+\_\+\+T\+H\+AT(my\+\_\+tuple, Fields\+Are(Ge(0), Has\+Substr(\char`\"{}hello\char`\"{})));

struct My\+Struct \{ int value = 42; std\+::string greeting = \char`\"{}aloha\char`\"{}; \}; My\+Struct s; E\+X\+P\+E\+C\+T\+\_\+\+T\+H\+AT(s, Fields\+Are(42, \char`\"{}aloha\char`\"{})); \`{}\`{}\`{}
\item Don\textquotesingle{}t use {\ttfamily Property()} against member functions that you do not own, because taking addresses of functions is fragile and generally not part of the contract of the function.
\end{DoxyItemize}

\subsection*{Matching the Result of a Function, Functor, or Callback}

\tabulinesep=1mm
\begin{longtabu} spread 0pt [c]{*{2}{|X[-1]}|}
\hline
\rowcolor{\tableheadbgcolor}\textbf{ Matcher  }&\textbf{ Description   }\\\cline{1-2}
\endfirsthead
\hline
\endfoot
\hline
\rowcolor{\tableheadbgcolor}\textbf{ Matcher  }&\textbf{ Description   }\\\cline{1-2}
\endhead
{\ttfamily Result\+Of(f, m)}  &{\ttfamily f(argument)} matches matcher {\ttfamily m}, where {\ttfamily f} is a function or functor.   \\\cline{1-2}
{\ttfamily Result\+Of(result\+\_\+description, f, m)}  &The same as the two-\/parameter version, but provides a better error message.   \\\cline{1-2}
\end{longtabu}


\subsection*{Pointer Matchers}

\tabulinesep=1mm
\begin{longtabu} spread 0pt [c]{*{2}{|X[-1]}|}
\hline
\rowcolor{\tableheadbgcolor}\textbf{ Matcher  }&\textbf{ Description   }\\\cline{1-2}
\endfirsthead
\hline
\endfoot
\hline
\rowcolor{\tableheadbgcolor}\textbf{ Matcher  }&\textbf{ Description   }\\\cline{1-2}
\endhead
{\ttfamily Address(m)}  &the result of {\ttfamily std\+::addressof(argument)} matches {\ttfamily m}.   \\\cline{1-2}
{\ttfamily Pointee(m)}  &{\ttfamily argument} (either a smart pointer or a raw pointer) points to a value that matches matcher {\ttfamily m}.   \\\cline{1-2}
{\ttfamily Pointer(m)}  &{\ttfamily argument} (either a smart pointer or a raw pointer) contains a pointer that matches {\ttfamily m}. {\ttfamily m} will match against the raw pointer regardless of the type of {\ttfamily argument}.   \\\cline{1-2}
{\ttfamily When\+Dynamic\+Cast\+To$<$T$>$(m)}  &when {\ttfamily argument} is passed through {\ttfamily dynamic\+\_\+cast$<$T$>$()}, it matches matcher {\ttfamily m}.   \\\cline{1-2}
\end{longtabu}


Technically, all matchers match a {\itshape single} value. A \char`\"{}multi-\/argument\char`\"{} matcher is just one that matches a {\itshape tuple}. The following matchers can be used to match a tuple {\ttfamily (x, y)}\+:

\tabulinesep=1mm
\begin{longtabu} spread 0pt [c]{*{2}{|X[-1]}|}
\hline
\rowcolor{\tableheadbgcolor}\textbf{ Matcher  }&\textbf{ Description   }\\\cline{1-2}
\endfirsthead
\hline
\endfoot
\hline
\rowcolor{\tableheadbgcolor}\textbf{ Matcher  }&\textbf{ Description   }\\\cline{1-2}
\endhead
{\ttfamily Eq()}  &{\ttfamily x == y}   \\\cline{1-2}
{\ttfamily Ge()}  &{\ttfamily x $>$= y}   \\\cline{1-2}
{\ttfamily Gt()}  &{\ttfamily x $>$ y}   \\\cline{1-2}
{\ttfamily Le()}  &{\ttfamily x $<$= y}   \\\cline{1-2}
{\ttfamily Lt()}  &{\ttfamily x $<$ y}   \\\cline{1-2}
{\ttfamily Ne()}  &{\ttfamily x != y}   \\\cline{1-2}
\end{longtabu}


You can use the following selectors to pick a subset of the arguments (or reorder them) to participate in the matching\+:

\tabulinesep=1mm
\begin{longtabu} spread 0pt [c]{*{2}{|X[-1]}|}
\hline
\rowcolor{\tableheadbgcolor}\textbf{ Matcher  }&\textbf{ Description   }\\\cline{1-2}
\endfirsthead
\hline
\endfoot
\hline
\rowcolor{\tableheadbgcolor}\textbf{ Matcher  }&\textbf{ Description   }\\\cline{1-2}
\endhead
{\ttfamily All\+Args(m)}  &Equivalent to {\ttfamily m}. Useful as syntactic sugar in {\ttfamily .With(\+All\+Args(m))}.   \\\cline{1-2}
{\ttfamily Args$<$N1, N2, ..., Nk$>$(m)}  &The tuple of the {\ttfamily k} selected (using 0-\/based indices) arguments matches {\ttfamily m}, e.\+g. {\ttfamily Args$<$1, 2$>$(Eq())}.   \\\cline{1-2}
\end{longtabu}


\subsection*{Composite Matchers}

You can make a matcher from one or more other matchers\+:

\tabulinesep=1mm
\begin{longtabu} spread 0pt [c]{*{2}{|X[-1]}|}
\hline
\rowcolor{\tableheadbgcolor}\textbf{ Matcher  }&\textbf{ Description   }\\\cline{1-2}
\endfirsthead
\hline
\endfoot
\hline
\rowcolor{\tableheadbgcolor}\textbf{ Matcher  }&\textbf{ Description   }\\\cline{1-2}
\endhead
{\ttfamily All\+Of(m1, m2, ..., mn)}  &{\ttfamily argument} matches all of the matchers {\ttfamily m1} to {\ttfamily mn}.   \\\cline{1-2}
{\ttfamily All\+Of\+Array(\{m0, m1, ..., mn\})}, {\ttfamily All\+Of\+Array(a\+\_\+container)}, {\ttfamily All\+Of\+Array(begin, end)}, {\ttfamily All\+Of\+Array(array)}, or {\ttfamily All\+Of\+Array(array, count)}  &The same as {\ttfamily All\+Of()} except that the matchers come from an initializer list, S\+T\+L-\/style container, iterator range, or C-\/style array.   \\\cline{1-2}
{\ttfamily Any\+Of(m1, m2, ..., mn)}  &{\ttfamily argument} matches at least one of the matchers {\ttfamily m1} to {\ttfamily mn}.   \\\cline{1-2}
{\ttfamily Any\+Of\+Array(\{m0, m1, ..., mn\})}, {\ttfamily Any\+Of\+Array(a\+\_\+container)}, {\ttfamily Any\+Of\+Array(begin, end)}, {\ttfamily Any\+Of\+Array(array)}, or {\ttfamily Any\+Of\+Array(array, count)}  &The same as {\ttfamily Any\+Of()} except that the matchers come from an initializer list, S\+T\+L-\/style container, iterator range, or C-\/style array.   \\\cline{1-2}
{\ttfamily Not(m)}  &{\ttfamily argument} doesn\textquotesingle{}t match matcher {\ttfamily m}.   \\\cline{1-2}
{\ttfamily Conditional(cond, m1, m2)}  &Matches matcher {\ttfamily m1} if {\ttfamily cond} evaluates to true, else matches {\ttfamily m2}.   \\\cline{1-2}
\end{longtabu}


\subsection*{Adapters for Matchers}

\tabulinesep=1mm
\begin{longtabu} spread 0pt [c]{*{2}{|X[-1]}|}
\hline
\rowcolor{\tableheadbgcolor}\textbf{ Matcher  }&\textbf{ Description   }\\\cline{1-2}
\endfirsthead
\hline
\endfoot
\hline
\rowcolor{\tableheadbgcolor}\textbf{ Matcher  }&\textbf{ Description   }\\\cline{1-2}
\endhead
{\ttfamily Matcher\+Cast$<$T$>$(m)}  &casts matcher {\ttfamily m} to type {\ttfamily Matcher$<$T$>$}.   \\\cline{1-2}
{\ttfamily Safe\+Matcher\+Cast$<$T$>$(m)}  &\href{../gmock_cook_book.md#SafeMatcherCast}{\tt safely casts} matcher {\ttfamily m} to type {\ttfamily Matcher$<$T$>$}.   \\\cline{1-2}
{\ttfamily Truly(predicate)}  &{\ttfamily predicate(argument)} returns something considered by C++ to be true, where {\ttfamily predicate} is a function or functor.   \\\cline{1-2}
\end{longtabu}


{\ttfamily Address\+Satisfies(callback)} and {\ttfamily Truly(callback)} take ownership of {\ttfamily callback}, which must be a permanent callback.

\tabulinesep=1mm
\begin{longtabu} spread 0pt [c]{*{2}{|X[-1]}|}
\hline
\rowcolor{\tableheadbgcolor}\textbf{ Matcher  }&\textbf{ Description   }\\\cline{1-2}
\endfirsthead
\hline
\endfoot
\hline
\rowcolor{\tableheadbgcolor}\textbf{ Matcher  }&\textbf{ Description   }\\\cline{1-2}
\endhead
{\ttfamily Matches(m)(value)}  &evaluates to {\ttfamily true} if {\ttfamily value} matches {\ttfamily m}. You can use {\ttfamily Matches(m)} alone as a unary functor.   \\\cline{1-2}
{\ttfamily Explain\+Match\+Result(m, value, result\+\_\+listener)}  &evaluates to {\ttfamily true} if {\ttfamily value} matches {\ttfamily m}, explaining the result to {\ttfamily result\+\_\+listener}.   \\\cline{1-2}
{\ttfamily Value(value, m)}  &evaluates to {\ttfamily true} if {\ttfamily value} matches {\ttfamily m}.   \\\cline{1-2}
\end{longtabu}


\subsection*{Defining Matchers}

\tabulinesep=1mm
\begin{longtabu} spread 0pt [c]{*{2}{|X[-1]}|}
\hline
\rowcolor{\tableheadbgcolor}\textbf{ Macro  }&\textbf{ Description   }\\\cline{1-2}
\endfirsthead
\hline
\endfoot
\hline
\rowcolor{\tableheadbgcolor}\textbf{ Macro  }&\textbf{ Description   }\\\cline{1-2}
\endhead
{\ttfamily M\+A\+T\+C\+H\+ER(Is\+Even, \char`\"{}\char`\"{}) \{ return (arg \% 2) == 0; \}}  &Defines a matcher {\ttfamily Is\+Even()} to match an even number.   \\\cline{1-2}
{\ttfamily M\+A\+T\+C\+H\+E\+R\+\_\+P(Is\+Divisible\+By, n, \char`\"{}\char`\"{}) \{ $\ast$result\+\_\+listener $<$$<$ \char`\"{}where the remainder is \char`\"{} $<$$<$ (arg \% n); return (arg \% n) == 0; \}}  &Defines a matcher {\ttfamily Is\+Divisible\+By(n)} to match a number divisible by {\ttfamily n}.   \\\cline{1-2}
{\ttfamily M\+A\+T\+C\+H\+E\+R\+\_\+\+P2(Is\+Between, a, b, absl\+::\+Str\+Cat(negation ? \char`\"{}isn\textquotesingle{}t\char`\"{} \+: \char`\"{}is\char`\"{}, \char`\"{} between \char`\"{}, Print\+To\+String(a), \char`\"{} and \char`\"{}, Print\+To\+String(b))) \{ return a $<$= arg \&\& arg $<$= b; \}}  &Defines a matcher {\ttfamily Is\+Between(a, b)} to match a value in the range \mbox{[}{\ttfamily a}, {\ttfamily b}\mbox{]}.   \\\cline{1-2}
\end{longtabu}


{\bfseries Notes\+:}


\begin{DoxyEnumerate}
\item The {\ttfamily M\+A\+T\+C\+H\+E\+R$\ast$} macros cannot be used inside a function or class.
\item The matcher body must be {\itshape purely functional} (i.\+e. it cannot have any side effect, and the result must not depend on anything other than the value being matched and the matcher parameters).
\item You can use {\ttfamily Print\+To\+String(x)} to convert a value {\ttfamily x} of any type to a string.
\item You can use {\ttfamily Explain\+Match\+Result()} in a custom matcher to wrap another matcher, for example\+:

\`{}\`{}\`{}cpp M\+A\+T\+C\+H\+E\+R\+\_\+P(Nested\+Property\+Matches, matcher, \char`\"{}\char`\"{}) \{ return Explain\+Match\+Result(matcher, arg.\+nested().property(), result\+\_\+listener); \} \`{}\`{}\`{}
\item You can use {\ttfamily Describe\+Matcher$<$$>$} to describe another matcher. For example\+:

\`{}\`{}\`{}cpp M\+A\+T\+C\+H\+E\+R\+\_\+P(X\+And\+Y\+That, matcher, \char`\"{}\+X that \char`\"{} + Describe\+Matcher$<$int$>$(matcher, negation) + (negation ? \char`\"{} or\char`\"{} \+: \char`\"{} and\char`\"{}) + \char`\"{} Y that \char`\"{} + Describe\+Matcher$<$double$>$(matcher, negation)) \{ return Explain\+Match\+Result(matcher, arg.\+x(), result\+\_\+listener) \&\& Explain\+Match\+Result(matcher, arg.\+y(), result\+\_\+listener); \} \`{}\`{}\`{} 
\end{DoxyEnumerate}