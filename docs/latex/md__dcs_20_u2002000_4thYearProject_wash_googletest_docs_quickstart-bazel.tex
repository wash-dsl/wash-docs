This tutorial aims to get you up and running with Google\+Test using the Bazel build system. If you\textquotesingle{}re using Google\+Test for the first time or need a refresher, we recommend this tutorial as a starting point.

\subsection*{Prerequisites}

To complete this tutorial, you\textquotesingle{}ll need\+:


\begin{DoxyItemize}
\item A compatible operating system (e.\+g. Linux, mac\+OS, Windows).
\item A compatible C++ compiler that supports at least C++14.
\item \href{https://bazel.build/}{\tt Bazel}, the preferred build system used by the Google\+Test team.
\end{DoxyItemize}

See Supported Platforms for more information about platforms compatible with Google\+Test.

If you don\textquotesingle{}t already have Bazel installed, see the \href{https://bazel.build/install}{\tt Bazel installation guide}.

\{\+: .callout .note\} Note\+: The terminal commands in this tutorial show a Unix shell prompt, but the commands work on the Windows command line as well.

\subsection*{Set up a Bazel workspace}

A \href{https://docs.bazel.build/versions/main/build-ref.html#workspace}{\tt Bazel workspace} is a directory on your filesystem that you use to manage source files for the software you want to build. Each workspace directory has a text file named {\ttfamily W\+O\+R\+K\+S\+P\+A\+CE} which may be empty, or may contain references to external dependencies required to build the outputs.

First, create a directory for your workspace\+:


\begin{DoxyCode}
$ mkdir my\_workspace && cd my\_workspace
\end{DoxyCode}


Next, you’ll create the {\ttfamily W\+O\+R\+K\+S\+P\+A\+CE} file to specify dependencies. A common and recommended way to depend on Google\+Test is to use a \href{https://docs.bazel.build/versions/main/external.html}{\tt Bazel external dependency} via the \href{https://docs.bazel.build/versions/main/repo/http.html#http_archive}{\tt {\ttfamily http\+\_\+archive} rule}. To do this, in the root directory of your workspace ({\ttfamily my\+\_\+workspace/}), create a file named {\ttfamily W\+O\+R\+K\+S\+P\+A\+CE} with the following contents\+:


\begin{DoxyCode}
load("@bazel\_tools//tools/build\_defs/repo:http.bzl", "http\_archive")

http\_archive(
  name = "com\_google\_googletest",
  urls = ["https://github.com/google/googletest/archive/5ab508a01f9eb089207ee87fd547d290da39d015.zip"],
  strip\_prefix = "googletest-5ab508a01f9eb089207ee87fd547d290da39d015",
)
\end{DoxyCode}


The above configuration declares a dependency on Google\+Test which is downloaded as a Z\+IP archive from Git\+Hub. In the above example, {\ttfamily 5ab508a01f9eb089207ee87fd547d290da39d015} is the Git commit hash of the Google\+Test version to use; we recommend updating the hash often to point to the latest version. Use a recent hash on the {\ttfamily main} branch.

Now you\textquotesingle{}re ready to build C++ code that uses Google\+Test.

\subsection*{Create and run a binary}

With your Bazel workspace set up, you can now use Google\+Test code within your own project.

As an example, create a file named {\ttfamily hello\+\_\+test.\+cc} in your {\ttfamily my\+\_\+workspace} directory with the following contents\+:


\begin{DoxyCode}
\textcolor{preprocessor}{#include <gtest/gtest.h>}

\textcolor{comment}{// Demonstrate some basic assertions.}
TEST(HelloTest, BasicAssertions) \{
  \textcolor{comment}{// Expect two strings not to be equal.}
  EXPECT\_STRNE(\textcolor{stringliteral}{"hello"}, \textcolor{stringliteral}{"world"});
  \textcolor{comment}{// Expect equality.}
  EXPECT\_EQ(7 * 6, 42);
\}
\end{DoxyCode}


Google\+Test provides \href{primer.md#assertions}{\tt assertions} that you use to test the behavior of your code. The above sample includes the main Google\+Test header file and demonstrates some basic assertions.

To build the code, create a file named {\ttfamily B\+U\+I\+LD} in the same directory with the following contents\+:


\begin{DoxyCode}
cc\_test(
  name = "hello\_test",
  size = "small",
  srcs = ["hello\_test.cc"],
  deps = ["@com\_google\_googletest//:gtest\_main"],
)
\end{DoxyCode}


This {\ttfamily cc\+\_\+test} rule declares the C++ test binary you want to build, and links to Google\+Test ({\ttfamily //\+:gtest\+\_\+main}) using the prefix you specified in the {\ttfamily W\+O\+R\+K\+S\+P\+A\+CE} file ({\ttfamily @com\+\_\+google\+\_\+googletest}). For more information about Bazel {\ttfamily B\+U\+I\+LD} files, see the \href{https://docs.bazel.build/versions/main/tutorial/cpp.html}{\tt Bazel C++ Tutorial}.

\{\+: .callout .note\} N\+O\+TE\+: In the example below, we assume Clang or G\+CC and set {\ttfamily -\/-\/cxxopt=-\/std=c++14} to ensure that Google\+Test is compiled as C++14 instead of the compiler\textquotesingle{}s default setting (which could be C++11). For M\+S\+VC, the equivalent would be {\ttfamily -\/-\/cxxopt=/std\+:c++14}. See Supported Platforms for more details on supported language versions.

Now you can build and run your test\+:


\begin{DoxyPre}
{\bfseries my\_workspace\$ bazel test --cxxopt=-std=c++14 --test\_output=all //:hello\_test}
INFO: Analyzed target //:hello\_test (26 packages loaded, 362 targets configured).
INFO: Found 1 test target...
INFO: From Testing //:hello\_test:
==================== Test output for //:hello\_test:
Running main() from gmock\_main.cc
[==========] Running 1 test from 1 test suite.
[----------] Global test environment set-up.
[----------] 1 test from HelloTest
[ RUN      ] HelloTest.BasicAssertions
[       OK ] HelloTest.BasicAssertions (0 ms)
[----------] 1 test from HelloTest (0 ms total)\end{DoxyPre}



\begin{DoxyPre}[----------] Global test environment tear-down
[==========] 1 test from 1 test suite ran. (0 ms total)
[  PASSED  ] 1 test.
================================================================================
Target //:hello\_test up-to-date:
  bazel-bin/hello\_test
INFO: Elapsed time: 4.190s, Critical Path: 3.05s
INFO: 27 processes: 8 internal, 19 linux-sandbox.
INFO: Build completed successfully, 27 total actions
//:hello\_test                                                     PASSED in 0.1s\end{DoxyPre}



\begin{DoxyPre}INFO: Build completed successfully, 27 total actions
\end{DoxyPre}


Congratulations! You\textquotesingle{}ve successfully built and run a test binary using Google\+Test.

\subsection*{Next steps}


\begin{DoxyItemize}
\item Check out the Primer to start learning how to write simple tests.
\item See the code samples for more examples showing how to use a variety of Google\+Test features. 
\end{DoxyItemize}