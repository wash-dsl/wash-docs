\hypertarget{write__hdf5_8cpp}{}\section{/dcs/20/u2002000/4th\+Year\+Project/wash/io/write\+\_\+hdf5.cpp File Reference}
\label{write__hdf5_8cpp}\index{/dcs/20/u2002000/4th\+Year\+Project/wash/io/write\+\_\+hdf5.\+cpp@{/dcs/20/u2002000/4th\+Year\+Project/wash/io/write\+\_\+hdf5.\+cpp}}


Write out the simulation state serially to a H\+D\+F5 file.  


{\ttfamily \#include \char`\"{}hdf5.\+hpp\char`\"{}}\newline


\subsection{Detailed Description}
Write out the simulation state serially to a H\+D\+F5 file. 

\begin{DoxyAuthor}{Author}
James Macer-\/\+Wright 
\end{DoxyAuthor}
\begin{DoxyVersion}{Version}
0.\+1 
\end{DoxyVersion}
\begin{DoxyDate}{Date}
2023-\/11-\/15
\end{DoxyDate}
\begin{DoxyCopyright}{Copyright}
Copyright (c) 2023
\end{DoxyCopyright}
Periodically we want to write out the simulation data to a file to inspect the intermediate values of particles \& properties. This can be achieved with a number of different backend technologies, H\+D\+F5 is one which supports parallelism with M\+PI.

As well, H\+D\+F5 formats can be read by visualisation software such as S\+P\+La\+SH which was built for visualising S\+PH simulation data.

We expect H\+D\+F5 to be built and present on the system for this use. 