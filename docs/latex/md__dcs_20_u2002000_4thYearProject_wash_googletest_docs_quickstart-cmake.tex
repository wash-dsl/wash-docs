This tutorial aims to get you up and running with Google\+Test using C\+Make. If you\textquotesingle{}re using Google\+Test for the first time or need a refresher, we recommend this tutorial as a starting point. If your project uses Bazel, see the Quickstart for Bazel instead.

\subsection*{Prerequisites}

To complete this tutorial, you\textquotesingle{}ll need\+:


\begin{DoxyItemize}
\item A compatible operating system (e.\+g. Linux, mac\+OS, Windows).
\item A compatible C++ compiler that supports at least C++14.
\item \href{https://cmake.org/}{\tt C\+Make} and a compatible build tool for building the project.
\begin{DoxyItemize}
\item Compatible build tools include \href{https://www.gnu.org/software/make/}{\tt Make}, \href{https://ninja-build.org/}{\tt Ninja}, and others -\/ see \href{https://cmake.org/cmake/help/latest/manual/cmake-generators.7.html}{\tt C\+Make Generators} for more information.
\end{DoxyItemize}
\end{DoxyItemize}

See Supported Platforms for more information about platforms compatible with Google\+Test.

If you don\textquotesingle{}t already have C\+Make installed, see the \href{https://cmake.org/install}{\tt C\+Make installation guide}.

\{\+: .callout .note\} Note\+: The terminal commands in this tutorial show a Unix shell prompt, but the commands work on the Windows command line as well.

\subsection*{Set up a project}

C\+Make uses a file named {\ttfamily C\+Make\+Lists.\+txt} to configure the build system for a project. You\textquotesingle{}ll use this file to set up your project and declare a dependency on Google\+Test.

First, create a directory for your project\+:


\begin{DoxyCode}
$ mkdir my\_project && cd my\_project
\end{DoxyCode}


Next, you\textquotesingle{}ll create the {\ttfamily C\+Make\+Lists.\+txt} file and declare a dependency on Google\+Test. There are many ways to express dependencies in the C\+Make ecosystem; in this quickstart, you\textquotesingle{}ll use the \href{https://cmake.org/cmake/help/latest/module/FetchContent.html}{\tt {\ttfamily Fetch\+Content} C\+Make module}. To do this, in your project directory ({\ttfamily my\+\_\+project}), create a file named {\ttfamily C\+Make\+Lists.\+txt} with the following contents\+:


\begin{DoxyCode}
cmake\_minimum\_required(VERSION 3.14)
project(my\_project)

# GoogleTest requires at least C++14
set(CMAKE\_CXX\_STANDARD 14)
set(CMAKE\_CXX\_STANDARD\_REQUIRED ON)

include(FetchContent)
FetchContent\_Declare(
  googletest
  URL https://github.com/google/googletest/archive/03597a01ee50ed33e9dfd640b249b4be3799d395.zip
)
# For Windows: Prevent overriding the parent project's compiler/linker settings
set(gtest\_force\_shared\_crt ON CACHE BOOL "" FORCE)
FetchContent\_MakeAvailable(googletest)
\end{DoxyCode}


The above configuration declares a dependency on Google\+Test which is downloaded from Git\+Hub. In the above example, {\ttfamily 03597a01ee50ed33e9dfd640b249b4be3799d395} is the Git commit hash of the Google\+Test version to use; we recommend updating the hash often to point to the latest version.

For more information about how to create {\ttfamily C\+Make\+Lists.\+txt} files, see the \href{https://cmake.org/cmake/help/latest/guide/tutorial/index.html}{\tt C\+Make Tutorial}.

\subsection*{Create and run a binary}

With Google\+Test declared as a dependency, you can use Google\+Test code within your own project.

As an example, create a file named {\ttfamily hello\+\_\+test.\+cc} in your {\ttfamily my\+\_\+project} directory with the following contents\+:


\begin{DoxyCode}
\textcolor{preprocessor}{#include <gtest/gtest.h>}

\textcolor{comment}{// Demonstrate some basic assertions.}
TEST(HelloTest, BasicAssertions) \{
  \textcolor{comment}{// Expect two strings not to be equal.}
  EXPECT\_STRNE(\textcolor{stringliteral}{"hello"}, \textcolor{stringliteral}{"world"});
  \textcolor{comment}{// Expect equality.}
  EXPECT\_EQ(7 * 6, 42);
\}
\end{DoxyCode}


Google\+Test provides \href{primer.md#assertions}{\tt assertions} that you use to test the behavior of your code. The above sample includes the main Google\+Test header file and demonstrates some basic assertions.

To build the code, add the following to the end of your {\ttfamily C\+Make\+Lists.\+txt} file\+:


\begin{DoxyCode}
enable\_testing()

add\_executable(
  hello\_test
  hello\_test.cc
)
target\_link\_libraries(
  hello\_test
  GTest::gtest\_main
)

include(GoogleTest)
gtest\_discover\_tests(hello\_test)
\end{DoxyCode}


The above configuration enables testing in C\+Make, declares the C++ test binary you want to build ({\ttfamily hello\+\_\+test}), and links it to Google\+Test ({\ttfamily gtest\+\_\+main}). The last two lines enable C\+Make\textquotesingle{}s test runner to discover the tests included in the binary, using the \href{https://cmake.org/cmake/help/git-stage/module/GoogleTest.html}{\tt {\ttfamily Google\+Test} C\+Make module}.

Now you can build and run your test\+:


\begin{DoxyPre}
{\bfseries my\_project\$ cmake -S . -B build}
-- The C compiler identification is GNU 10.2.1
-- The CXX compiler identification is GNU 10.2.1
...
-- Build files have been written to: .../my\_project/build\end{DoxyPre}



\begin{DoxyPre}{\bfseries my\_project\$ cmake --build build}
Scanning dependencies of target gtest
...
[100\%] Built target gmock\_main\end{DoxyPre}



\begin{DoxyPre}{\bfseries my\_project\$ cd build \&\& ctest}
Test project .../my\_project/build
    Start 1: HelloTest.BasicAssertions
1/1 Test \#1: HelloTest.BasicAssertions ........   Passed    0.00 sec\end{DoxyPre}



\begin{DoxyPre}100\% tests passed, 0 tests failed out of 1\end{DoxyPre}



\begin{DoxyPre}Total Test time (real) =   0.01 sec
\end{DoxyPre}


Congratulations! You\textquotesingle{}ve successfully built and run a test binary using Google\+Test.

\subsection*{Next steps}


\begin{DoxyItemize}
\item Check out the Primer to start learning how to write simple tests.
\item See the code samples for more examples showing how to use a variety of Google\+Test features. 
\end{DoxyItemize}